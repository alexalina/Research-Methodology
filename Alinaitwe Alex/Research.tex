\documentclass[10pt,a4paper]{article}
\usepackage[utf8]{inputenc}
\usepackage{amsmath}
\usepackage{amsfonts}
\usepackage{amssymb}
\author{ALINAITWE}
\begin{document}
\begin{center}
\textbf{MAKERERE UNIVERSITY} \newline
\textbf{COLLEGE OF COMPUTING AND INFORMATION SCIENCE} \newline
\textbf{SCHOOL OF COMPUTING AND INFORMATICS TECHNOLOGY} \newline
\textbf{BIT 2207 RESEARCH METHODOLOGY} \newline
\textbf{INDIVIDUAL RESEARCH WORK} \newline
\end{center}
\begin{flushleft}
\textbf{INSTRUCTOR:} Mr. ERNEST MWEBAZE \newline
\textbf{NAME:} ALEX ALINAITWE \newline
\textbf{STD NO:} 216000284 \newline
\textbf{REG NO:} 16/U/81 \newline
\end{flushleft}
\begin{center}
\textit{QUESTION:} \newline
\textbf{WHAT ARE THE CAUSES OF YOUTH UNEMPLOYMENT IN DEVELOPING COUNTRIES?}
\end{center}
\begin{flushleft}
In today's modern global economy, unemployment is a serious issue that affects the livelihoods
and qualities of life of an estimated 8.4 percent of the world's lobar force. Youth unemployment in
particular is the most damaging type of unemployment, as it affects young adults under 25 who
are just beginning to enter the workforce, and can leave a serious impact on a person's ability to
find and retain work later in life. Young people make up 40 percent of all the unemployed people in
the world. Up to two-thirds of working age Youth unemployment remains a serious policy challenge in many sub-Saharan African countries, including Uganda. In 2013, youth (aged 15 to 24) in sub-Saharan Africa were twice likely to be unemployed compared to any other age cohort. For Uganda, in 2012, the Uganda Bureau of Statistics revealed that the share of unemployed youth (national definition, 18-30 years) among the total unemployed persons in the country was 64 percent. Given the rapid growth of the Ugandan, Three-quarters of the population are below the age of 30 years coupled with the fact that the youth are getting better educated through higher access to primary and secondary education, a stronger focus on job creation for this cohort of people cannot be overemphasized.\newline

Unemployment is a phenomenon that occurs when a person who is actively searching for employment is unable to find work. Unemployment is often used as a measure of the health of the economy. The most frequently measure of unemployment is the unemployment rate, which is the number of unemployed people divided by the number of people in the labour force.\newline

Causes of youth unemployment are believed to be multifaceted, ranging from an inadequate investment/supply side of jobs, insufficient employable skill(for instance youth possess skills that are not compatible with available jobs) and the high rates of labour force growth at 4.7 percent per annum. \newline

Additionally, high population growth in regions such as North and Sub-Saharan Africa have led to a larger proportion of youth entering the labour market, contributing to higher rates of youth unemployment. \newline

Corruption, This is yet another reason as to why the youth in developing countries are still unemployed. Funds meant for development projects have been misappropriated, diverted, or embezzled and stashed away in foreign banks, while some incompetent and corrupt bureaucrats and administrators in the public enterprise have liquidated these organizations. The point being made here is that the collaboration of the political elites, local and foreign contractors in the inflation of contract fees have robbed developing countries a chance to develop a vibrant economy that would have created jobs for the youths in various sectors of the economy. \newline

The rapid expansion of the education system, this has led to increase in the supply of educated manpower above the corresponding demand for them, this  contributes to the problem of the youth unemployment. Ordinarily the increasing number of higher institutions in developing countries is not a problem, but the reality is that the economies of these developing countries are to weak to absorb the large number of graduates they produce every year. \newline 
\end{flushleft}
\paragraph{This is \textbf{Analytical Research} since there are facts in place that can be based on to make evaluations and come up with appropriate solutions to address the problem of youth unemployment.}

\end{document}